\documentclass[]{article}
\usepackage{ctex,hyperref}% 输出汉字
\usepackage{amsmath,amssymb,amsfonts}
\usepackage{amsthm,amsmath,amssymb}
\usepackage{mathrsfs}
%opening
\usepackage{setspace}
\usepackage{lipsum}
\usepackage{graphicx}% 图片插入宏包
\usepackage{subfigure}% 并排子图
\usepackage{float}% 浮动环境,用于调整图片位置
\usepackage[export]{adjustbox}% 防止过宽的图片
\usepackage{amsmath}
\usepackage{extarrows}
\graphicspath{{Fig/}}%文章所用图片在当前目录下的 Figures目录
\usepackage{listings}
\usepackage{xcolor} % 用于自定义代码颜色
\lstset{
	language=Python,            % 设置语言为 Python
	basicstyle=\ttfamily,       % 代码字体为等宽字体
	keywordstyle=\color{blue},  % 关键字颜色
	stringstyle=\color{red},    % 字符串颜色
	commentstyle=\color{gray},  % 注释颜色
	numbers=left,               % 在左侧显示行号
	numberstyle=\tiny\color{gray}, % 行号样式
	stepnumber=1,               % 行号的间隔
	showspaces=false,           % 不显示空格符
	showstringspaces=false,     % 不在字符串中显示空格
	frame=single,               % 用方框包围代码
	breaklines=true             % 自动换行
}

\title{FBR and DVR}
\author{步允霆}

\begin{document}
	\maketitle
	
\section{有限基组表象(FBR)}
在一维问题中,哈密顿算符为
\begin{equation}
	H=-\dfrac{\hbar^2}{2m}\dfrac{\mathrm{d}^2}{\mathrm{d}x^2}+V(x)
\end{equation}
将波函数$\Psi(x)$展开
\begin{equation}
	\Psi(x)=\sum\limits_nc_n\phi_n(x)
\end{equation}
在量子化学中,基函数通常采用高斯函数,这种基函数是一种不正交也不归一,在FBR下,我们可以选择一组正交归一的完备基,比如在一维势箱问题中,在势箱范围内势能为0,其解的sin型波函数则为动能的本征矢:
\begin{equation}
	|\phi_n\rangle=\sqrt{\dfrac{2}{b-a}}\mathrm{sin}\dfrac{n\pi(x-a)}{b-a}
\end{equation}
其中$a, b$为选择的区间,用此函数展开,我们有(为了书写方便,只采用脚标$n, m$表示$|\phi_n\rangle$)
\begin{align}
	&H|\Psi\rangle=E\Psi \rangle\nonumber\\
	&\sum\limits_nc_nH|n\rangle=\sum\limits_nc_nE|n\rangle
\end{align}
左乘$\langle m|$有
\begin{equation}
	\sum\limits_nc_n\langle m|H|n\rangle=\sum\limits_nc_nE\langle m|n\rangle
\end{equation}
共有$m$个这种方程,因此可以写作矩阵形式
\begin{equation}
	\mathbf{HC}=\mathbf{EC}
\end{equation}
对角化后,可以获得本征矢与本征值.

我们研究体系为$m=1, \omega=1$的谐振子,势能表达式为
\begin{equation}
	V(x)=\dfrac{1}{2}x^2
\end{equation}
在此例中,势能矩阵元$\langle m|V(x)|n\rangle$是可解析的,但一般情况下,我们无法解析或者很解析表达式很困难,于是我们可以采用高斯积分来计算矩阵元:
\begin{equation}
	\int f(x)\rho(x)\mathrm{d}x=\sum\limits_lA_lf(x_l)
\end{equation}
其中,$\rho(x)$为权重函数,$A_l$为权重,或者写作$w_l$,$x_l$是积分格点,对于我们选择的基函数,权重与格点为
\begin{equation}
	w_l=\dfrac{b-a}{N+1}
\end{equation}
\begin{equation}
	x_l=a+\dfrac{l(b-a)}{N+1}
\end{equation}
$N$为设定的格点数,因此,势能矩阵元为
\begin{equation}
	\langle m|V(x)|n\rangle=\sum\limits_lw_l\phi_m^*(x_l)V(x)\phi_n(x_l)
\end{equation}
由于我们选择的基函数是动能算符的本征矢,因此动能算符是对角的,其值可以简单的计算:
\begin{equation}
	\langle m|T|n\rangle=\dfrac{\hbar^2}{2m}\left( \dfrac{n\pi}{b-a}\right)^2\delta_{mn} 
\end{equation}
最后,在FBR表象下,波函数为
\begin{equation}
	|\Psi_i\rangle=\sum\limits_nc_{ni}|\phi_n\rangle
\end{equation}
\section{离散变量表示(FBR)}
FBR下计算动能矩阵十分方便,但是我们的难点一直是势能矩阵元的计算,由于我们使用高斯积分计算矩阵元,因此计算每一个矩阵元都要进行一次高斯积分,在势能形式非常复杂的情况下这是十分耗时且不能接受的,而我们的动能算符在不同体系下的形式都是一样的,因此我们可以采取一种表象变换,使得势能矩阵为对角矩阵。

对于势能矩阵元,我们有
\begin{equation}
\langle m|V(x)|n\rangle=\sum\limits_lw_l\phi_m^*(x_l)V(x)\phi_n(x_l)	
\end{equation}
令变换矩阵为
\begin{equation}
	B_{nl}=\sqrt{w_l}\phi_m(x_l)
\end{equation}
此时势能矩阵元转变为
\begin{equation}
	V_{mn}=\sum\limits_lB_{nl}V(x_l)B_{ml}
\end{equation}
写为矩阵形式
\begin{equation}
	\mathbf{V}=\mathbf{B\tilde{V}B^\dagger}
\end{equation}
其中
\begin{equation}
	\tilde{V}_{l^\prime l}=V(x_l)\delta_{l^\prime l}
\end{equation}
这样就实现了势能矩阵的对角化。于是
\begin{equation}
	(\mathbf{T}+\mathbf{B\tilde{V}B^\dagger})\mathbf{C}=\mathbf{EC}
\end{equation}
左乘$\mathbf{B^\dagger}$
\begin{align}
	&(\mathbf{B^\dagger}\mathbf{T}+\mathbf{B^\dagger}\mathbf{B\tilde{V}B^\dagger})\mathbf{C}=\mathbf{E}\mathbf{B^\dagger C}\nonumber\\
	&\mathbf{B^\dagger TB+\tilde{V}}\mathbf{B^\dagger C}=\mathbf{EB^\dagger C}
\end{align}
令$\mathbf{B^\dagger TB}=\mathbf{\tilde{T}}, \mathbf{B^\dagger C}=\mathbf{\tilde{C}}$,则
\begin{equation}
	(\mathbf{\tilde{T}+\tilde{V}})\mathbf{\tilde{C}}=\mathbf{E\tilde{C}}
\end{equation}
动能矩阵元可以解析而得
\begin{align}
	\tilde{T}_{i,j}&=\dfrac{1}{2m}\dfrac{\pi^2}{2(b-a)^2}(-1)^{i-j}\left[ \dfrac{1}{\mathrm{sin}^2[\pi (i-j)/2(N+1)]}-\dfrac{1}{\mathrm{sin}^2[\pi (i+j)/2(N+1)]}\right] \quad i\neq j \nonumber\\
	\tilde{T}_{i,j}&=\dfrac{1}{2m}\dfrac{\pi^2}{2(b-a)^2}\left[ \dfrac{2(N+1)^2+1}{3}-\dfrac{1}{\mathrm{sin}^2[\pi i/(N+1)]}\right] \quad i=j
\end{align}
DVR下,波函数为
\begin{equation}
	|\Psi_i\rangle=\sum\limits_l\tilde{c}_{li}|x_l\rangle
\end{equation}
其中
\begin{equation}
	\tilde{c}_{li}=\langle x_l|\Psi_i\rangle=\sqrt{w_l}\Psi(x_l)
\end{equation}
代入可以发现,DVR下的波函数矩阵就是系数矩阵。做出谐振子$n=0, n=1, n=2, n=15$四个能级的波函数如图
\begin{figure}[H]
	\centering
	\includegraphics[scale=0.6]{FBR.png}
	\caption{FBR波函数}
	\label{Figure 1}
\end{figure}
\begin{figure}[H]
	\centering
	\includegraphics[scale=0.6]{DVR.png}
	\caption{DVR波函数}
	\label{Figure 2}
\end{figure}
\section{变换矩阵与坐标矩阵}
变换矩阵有两种求法,第一种是直接解析
\begin{equation}
	B_{nl}=\langle n|x_l\rangle=\sqrt{\dfrac{2}{N+1}}\mathrm{sin}\left( \dfrac{nl\pi}{N+1}\right) 
\end{equation}
另一种是对角化坐标矩阵$\langle m|X|n\rangle$,其本征矢为$B_{nl}$,本征矢为$x_l$,如果使用高斯积分求解,两者数值相等,如果使用解析求解,会有一定的差异,笔者计算的解析积分为:
\begin{align}
	X_{nn}&=\dfrac{1}{2}(a^2-b^2)\quad&m=n\\
	X_{mn}&=4mn(b-a)\dfrac{-1+\mathrm{cos}(m\pi)\mathrm{cos}(n\pi)}{(m^2-n^2)^2\pi^2}\quad &m\neq n
\end{align}
读者可以自行验证其正确性。
\end{document}
